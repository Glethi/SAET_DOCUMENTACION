\section{Plan de Desarrollo}
Para poder realizar una planificación de este proyecto, partiremos del Ciclo de Vida del Software, el cual nos indica cada una de las etapas y el proceso que se debe de hacer en cada una para poder llevar a cabo el desarrollo de un software sobre cualquier índole. 
\\ \\
La primera de ellas es, desde luego, la \textbf{planificación}. En esta etapa vamos a obtener todos los requerimientos tanto funcionales (aquellos que son vitales para cumplir el propósito del desarrollo software) y los no funcionales (aquellos que no afectan directamente la funcionalidad del sistema, pero que se deben de tomar en cuenta). Dichos requerimientos se pueden obtener de varias formas, ya sea por medio de entrevistas o cuestionarios aplicados al cliente (usuario final). Se tocan temas como presupuestos, intervalos de tiempo para entregas y reuniones que lleguen a ser necesarias.
\\ \\
 La segunda etapa es el \textbf{análisis}. Una vez que se obtengan los requerimientos, pasaremos a esta etapa en la cual se diseñará el software desde un punto de vista procesos y lógica en base a los requerimientos. Para poder dar a entender tanto la parte diseño como el funcionamiento que tendrá el SAET, se utilizarán diagramas UML: Base de Datos, Actividades, Secuencia. En cada uno de ellos se pretende plasmar de manera muy clara y desglosada cada una de las tareas que el usuario final podrá realizar e interactuar con el software.
 \\ \\
 La tercera etapa es la \textbf{implementación, pruebas y documentación}. En esta fase del proyecto, se comienza a realizar el diseño de interfaces y la codificación tanto del 'frontend' (la parte visual, interacción con el usuario) y la parte 'backend' (toda la lógica de datos y la conexión con el modelo - base de datos). En cuanto a las pruebas se planea hacer de forma modular, es decir, cada uno de los módulos será probado una vez terminada su implementación. Una vez que las pruebas den un resultado positivo en ese módulo, se continuará con el siguiente que se tiene planeado. Por último, la documentación (como el este documento) donde se plasma todo lo que se ha realizado para el correcto desarrollo del software. 
 \\ \\
 La última etapa es el \textbf{despliegue y mantenimiento}. Una vez que el software se haya sometido a diversos tipos de pruebas y las haya pasado todas de manera exitosa, comienza la etapa del despliegue. De acuerdo con las tecnologías utilizadas para el desarrollo del software será la infraestructura a elegir para alojar todo lo necesario para que la herramienta pueda ser utilizada. Tanto la parte frontend como backend y la base de datos estarán alojadas en la nube (privada o pública). Para el caso del mantenimiento, se consideran mejoras que el cliente requiere (sobre todo en los primeros días de puesta en producción) así como corrección de algunos errores que sean encontrados. 
 \\ \\
 Cabe señalar que la \textbf{capacitación al usuario} también es muy importante, aunque no se tome en cuenta dentro del Ciclo de Vida del Software. Puede darse de manera presencial o como es en este caso, un manual de usuario para expresar de manera gráfica y escrita como es que funciona el SAET.