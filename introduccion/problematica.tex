\section{Problemática}
Diariamente se realizan miles de transacciones electrónicas en México, estas pasan por cierto tratamiento para poder ser evaluadas y procesadas correctamente, sin embargo, algunas veces existen discrepancias entre el banco emisor y el receptor de la trama. Para que una transacción sea evaluada como 'buena', se deben de revisar minuciosamente cada uno de los campos y subcampos que contiene dicha trama, esto depende de muchos factores y valores como son: medio de acceso, código de respuesta, entry mode, etc.
\\ \\
Si este procedimiento se realiza con algún tercero, el costo de horas hombre y de dinero puede llegar a ser muy elevado, además de las constantes 'revalidaciones' de las tramas cada vez que estas se corrigen.   