\begin{abstract}
	Este documento contiene toda la especificación y documentación a detalle del Sistema de Administración y Estadística Transaccional Tokenizado (SAET, por sus siglas). Se presenta el análisis de la problemática que pretendemos resolver con la implementación de este software, así como el diseño y la construcción del propio software.
	\\ \\
	El SAET es una herramienta que le permite a una empresa bancaria visualizar un análisis a profundidad de las transacciones que se realizan en cierto intervalo de tiempo y así, poder descubrir discrepancias entre los emisores y receptores de dichas transacciones.
	\\ \\
	Para el análisis y diseño de la aplicación, se parte desde el Modelo de Arquitectura 4 + 1 diseñado por Philippe Kruchten, el cual nos permite representar de manera gráfica y descriptiva el software desde diferentes puntos de vista: procesos, lógica, desarrollo, física y escenarios. Además, nos apoyaremos con Historias de Usuario para el análisis de la aplicación así como para dar a conocer como funciona el SAET.
\end{abstract}