\section{Requerimientos Funcionales}
En base a la problemática que se ha planteado, se realiza una lista de los requerimientos funcionales del SAET, para una mejor identificación de dichos requerimientos se emplea una nomenclatura especial: \textbf{RF'X'}, siendo \textbf{'X'} el número de identificación del requerimiento. 
\begin{itemize}
	\item \textbf{RF1:} El SAET dará acceso solo si se ingresan correctamente las credenciales solicitadas (nombre de usuario y contraseña), esto por medio de una interfaz gráfica.
	\item \textbf{RF2:} El SAET manejará dos tipos de roles del usuario: operador y administrador, cada uno de estos tendrá acceso a diferentes opciones de acuerdo al rol que pertenezca.
	\item \textbf{RF3:} El SAET permitirá al administrador y operador visualizar la información de los siguientes token's: 
		\subitem Medio de Acceso.
		\subitem Código de Respuesta.
		\subitem Entry Mode
		\subitem Token C4
		\subitem Token C0
		\subitem Token B3
		\subitem Token B4
		\subitem Token B2
	\\ \\
	Mostrando una pantalla por cada uno de estos token's, en las cuales se pueda visualizar toda la información proveniente del backend.
	\item  \textbf{RF4:} El SAET permitirá al administrador, gestionar a los usuarios que puedan acceder al sistema, es decir, podrán visualizar a todos los usuarios que ya se encuentren registrados en la base de datos, registrar un nuevo usuario y además actualizar o eliminar el registro de uno o varios usuarios existentes.
	\item  \textbf{RF5:} El SAET deberá validar toda la información solicitada en algún formulario, en caso de ser errónea, el sistema mostrará un mensaje de alerta. 
	\item \textbf{RF6:} El SAET deberá proporcionar la interacción necesaria (filtros, gráficas y tablas) para que el usuario obtenga la información que desea en ese momento. 
	\item \textbf{RF7:} El SAET permitirá en cada uno de los filtros, elegir diversos valores al mismo tiempo, los cuales deberán funcionar para cada una de las visualizaciones en todos los token's.
	\item  \textbf{RF8:} El SAET deberá validar todos los datos de los campos obtenidos por medio del 'backend' con la conexión a el modelo - base de datos. Esta validación se hará dependiendo principalmente de los siguientes token's: Medio de Acceso, Código de Respuesta y Entry Mode así como de los filtros que el usuario haya elegido.
	\item \textbf{RF9:} El SAET permitirá al usuairo cerrar sesión para mantener la seguridad e integridad de su cuenta.
\end{itemize} 