\section{Requerimientos No Funcionales}
Como ya se mencionó anteriormente, estos requerimientos no influyen de manera directa con la funcionalidad del sistema, sin embargo, es importante tomarlos en cuenta para una mejor aceptación y calidad en el software. De igual manera, se emplea una nomenclatura especial \textbf{RNF'X'}, siendo la \textbf{'X'} el número de requerimiento a describir.
\begin{itemize}
	\item \textbf{RNF1:} La interfaz gráfica de usuario (GUI por sus siglas en inglés), debe de estar bien definida y diseñada para cada una de las pantallas.
	\item \textbf{RNF2:} El tiempo de respuesta del SAET no será mayor a 200 milisegundos por petición realizada, con esto nos referimos a ingreso de datos o interacción con los filtros, tablas y gráficas.
	\item \textbf{RNF3:} El desarrollo del sistema se hará siguiendo el patrón de diseño 'Modelo-Vista-Controlador', además de seguir las buenas prácticas de la programación.
	\item \textbf{RNF4:} Toda la información ingresada al sistema no será guardada, a excepción de los registros de usuarios.
	\item \textbf{RNF5:} El SAET proporcionará mensajes de error que sean informativos para el usuario y ayudarlo a resolver el problema que se presenta.
	\item \textbf{RNF6:} El SAET tendrá una disponibilidad del 99\% de las veces que el usuario requiera acceder a él.
	\item \textbf{RNF7:} El SAET será una herramienta web, es decir, que se podrá acceder desde un navegador web siempre y cuando se tenga una conexión a internet.
\end{itemize}